\documentclass[11pt,a4paper]{article}

\usepackage[margin=0.7in]{geometry}

\usepackage[utf8]{inputenc}
\usepackage{graphicx}
\usepackage{microtype}
\usepackage{float}
\usepackage{rotating}
\usepackage{xcolor}
\usepackage{sectsty}
\usepackage{url}
\usepackage{hyperref}

\usepackage{minted}

\hypersetup{
    colorlinks=true,
    linkcolor=blue,
    filecolor=magenta,
    urlcolor=blue,
}

\urlstyle{same}
\definecolor{darkgray}{rgb}{0.66, 0.66, 0.66}
	\definecolor{blue(ryb)}{rgb}{0.01, 0.28, 1.0}
\definecolor{dimgray}{rgb}{0.41, 0.41, 0.41}
\sectionfont{\color{cyan}} 
% \subsectionfont{\color{blue}}



\renewcommand{\texttt}[1]{%
  \begingroup
  \ttfamily
  \begingroup\lccode`~=`/\lowercase{\endgroup\def~}{/\discretionary{}{}{}}%
  \begingroup\lccode`~=`[\lowercase{\endgroup\def~}{[\discretionary{}{}{}}%
  \begingroup\lccode`~=`.\lowercase{\endgroup\def~}{.\discretionary{}{}{}}%
  \catcode`/=\active\catcode`[=\active\catcode`.=\active
  \scantokens{#1\noexpand}%
  \endgroup
}

\title{SOEN6461 - Software Design Methodology\\
Design Project\\
Group 9 - Deliverable 3\\
\bigskip
\large{\centerline{\textbf{Professor Yann-Gaël Guéhéneuc}}}
}

\begin{document}

\date{}
\maketitle



% \bigskip
% \bigskip

\begin{table}[H]
\centering
\begin{tabular}{|l|l|}
\hline
\textbf{Student ID} & \textbf{Unique email}               \\ \hline
40124288   & plablisenter@outlook.com   \\ \hline
40089008   & user.40089008@gmail.com    \\ \hline
40091878   & SOEN6461cheraghi@gmail.com \\ \hline
40130791   & komal3194p@gmail.com       \\ \hline
\end{tabular}
\end{table}

% \vspace{25em}

% \centerline{Department of Computer Science and Software Engineering}

% \centerline{Gina Cody School of Engineering and Computer Science}

% \centerline{Concordia University}

% \clearpage

\section{Analysis and Design Requirements}

On the one hand, the PADL meta-model allows describing models of (object-oriented) programs.

On the other hand, “PlantUML is a component that allows to quickly write : […] class diagrams” (See \url{https://plantuml.com/}).

Analyze, design, and implement a Visitor that generates a textual description of any PADL model. The description should include \texttt{padl.kernel.IFirstClassEntity} (and possibly all \texttt{padl.kernel.IEntity}) as nodes. The arcs between nodes should describes the different relations between entities (see \texttt{padl.kernel.IRelationship} and its sub-interfaces). The description should conform to the syntax and semantics of \url{https://plantuml.com/classdiagram} so that PlantUML can be called on this description. Demonstrate your implementation by providing textual description of some PADL model as well as the corresponding PlantUML-generated class diagrams.

\section{Implementation}

\subsection{Create PADL model}

The PADL library contains an interface called \texttt{padl.kernel.ICodeLevelModel}. The implementation class of this interface capable of holding the whole PADL meta-model of a set of related classes.

To populate the meta-models, we use the \texttt{create(final ICodeLevelModelCreator aCodeLevelModelCreator)} method, with combination with the \texttt{padl.creator.classfile.CompleteClassFileCreator} class. The source code for creating meta-model is already provided by professor Yann-Gaël Guéhéneuc. We put it inside a singleton class \texttt{plantumlgenerator.utils.MetaModelCreator}. Since the provided source code is just a code snippet, many libraries are required to make it run.

\subsection{Resolve dependencies}

Here is the list of dependencies that are missing/required
\begin{itemize}
    \item package \texttt{padl.kernel} for various interfaces and classes of meta-model.
    \item package \texttt{padl.analysis} for \texttt{UnsupportedSourceModelException}
    \item package \texttt{padl.creator.classfile} for class \texttt{CompleteClassFileCreator}.
    \item package \texttt{padl.statement.creator.classfiles} for using the class \texttt{ConditionalModelAnnotator} and \texttt{LOCModelAnnotator}
    \item package \texttt{padl.util} for using the class \texttt{ModelStatistics}
\end{itemize}

Each of these package have an corresponding project in \url{https://github.com/ptidejteam/v5.2} repository. Each project also have some dependencies on other libraries as well. To allow separation of each project, we convert each corresponding project into a Maven project and use Maven package command (\texttt{mvn package}) to create jar files.

Finally, we all all the jar files to \texttt{libs} folder of the main project, then add the dependency in Maven's \texttt{pom.xml} file with \texttt{<scope>} attribute set to \texttt{system}.

The dependency in \texttt{pom.xml} file should look like this and repeated with appropriate for every required libraries:

\begin{minted}[mathescape,
               linenos,
               numbersep=5pt,
               gobble=2,
               frame=lines,
               framesep=2mm]{xml}
    <dependency>
      <groupId>padl-lib</groupId>
      <artifactId>padl-lib</artifactId>
      <version>5.2</version>
      <scope>system</scope>
      <systemPath>${basedir}/libs/padl.jar</systemPath>
    </dependency>
\end{minted}

For the two package \texttt{padl.creator.classfile} and \texttt{padl.statement.creator.classfiles}, we must use the source code (not jar files) to prevent exception of type \texttt{java.lang.NoClassDefFoundError} when trying to load a folder of java compiled \texttt{.class} files. This make the project must be develop using Eclipse IDE. In the future, we will make the project IDE independent using Gradle or Maven.

\subsection{Implement the \texttt{padl.visitor.IGenerator} interface}

We decided to implement the \texttt{padl.visitor.IGenerator} interface for the job, because this interface provide the method \texttt{getCode();}. The \texttt{padl.visitor.IWalker} provide the method \texttt{Object getResult();}, which is not really fit for the task we are trying to do.



\end{document}
